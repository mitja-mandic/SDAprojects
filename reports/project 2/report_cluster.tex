\documentclass[12pt]{article}
\usepackage{graphicx}
\usepackage{booktabs}
\usepackage[font=footnotesize,skip=5pt]{caption}
\usepackage[font=scriptsize,skip=0pt]{subcaption}
\usepackage{amsmath}
\usepackage{amsfonts}
\usepackage{amssymb}
\usepackage{lscape}
\usepackage{psfrag}
\usepackage[usenames]{color}
\usepackage{bbm}
\usepackage[update]{epstopdf}
\usepackage[bookmarks,pdfstartview=FitH,a4paper,pdfborder={0 0 0}]{hyperref}
\usepackage{verbatim}
\usepackage{listings}
\usepackage{textcomp}
\usepackage{course}
\usepackage{fancyhdr}
\usepackage{multirow}
\pagestyle{fancy}
\usepackage{tikz}
\usepackage{bm}
\usepackage{float}
%\usepackage{subfig}

\renewcommand{\sectionmark}[1]{\markboth{#1}{#1}}
\renewcommand{\subsectionmark}[1]{\markright{#1}}

\newtheorem{theorem}{Theorem}
\DeclareMathOperator{\Var}{Var}
\DeclareMathOperator{\Bias}{Bias}

\graphicspath{ {../../Images/} }
\usepackage[backend=biber, style=bwl-FU, sorting=nyt]{biblatex}
\addbibresource{bib.bib}

\fancyhf{}
\fancyhead[RO]{\nouppercase{\footnotesize\sc\leftmark\ \hrulefill\ \thepage}}
%\fancyhead[RE]{\nouppercase{\footnotesize\sc\thepage\ \hrulefill\ }}
\renewcommand{\headrulewidth}{0pt}

\makeatletter
\def\cleardoublepage{\clearpage\if@twoside \ifodd\c@page\else%
\hbox{}%
\thispagestyle{empty}%
\clearpage%
\if@twocolumn\hbox{}\clearpage\fi\fi\fi}
\makeatother


\renewcommand{\topfraction}{0.9}  % max fraction of floats at top
\renewcommand{\bottomfraction}{0.8} % max fraction of floats at bottom
% Parameters for TEXT pages (not float pages):
\setcounter{topnumber}{2}
\setcounter{bottomnumber}{2}
\setcounter{totalnumber}{4}            % 2 may work better
\setcounter{dbltopnumber}{2}           % for 2-column pages
\renewcommand{\dbltopfraction}{0.9}    % fit big float above 2-col. text
\renewcommand{\textfraction}{0.07}     % allow minimal text w. figs
% Parameters for FLOAT pages (not text pages):
\renewcommand{\floatpagefraction}{0.7}  % require fuller float pages
% N.B.: floatpagefraction MUST be less than topfraction !!
\renewcommand{\dblfloatpagefraction}{0.7} % require fuller float pages

\sloppy

\widowpenalty=10000
\clubpenalty=10000

\edef\today{%\number\day\
\ifcase\month\or
January\or February\or March\or April\or May\or June\or July\or
August\or September\or October\or November\or December\fi\ \number\year}
\title{\vspace*{40.0mm}
  \bf Report on project about clustering
         \vspace*{20.0mm} \\
  %\vspace{-20mm}\framebox{DRAFT VERSION}\vspace{20mm} \\
  \Large\bf Statistical Data Analysis 
  
 
  
  Project 2 \vspace*{20.0mm}
  \vspace*{40.0mm}}
\author{Mitja Mandić}
\date{ May 2022}

\begin{document}

\begin{figure}
  \parbox[t]{125mm}{
    \vspace*{6mm}
    \scriptsize\sf           DEPARTMENT OF MATHEMATICS \\
    \scriptsize\sf           FACULTY OF SCIENCE\\
    \scriptsize\sf           KU LEUVEN}
  \parbox[t]{40mm}{
    \begin{flushright}
      \includegraphics[height=15mm]{../images/logo.eps.pdf}
    \end{flushright}}
\end{figure}

\maketitle
\thispagestyle{empty}
\raggedbottom

\cleardoublepage
\pagenumbering{roman}
\setcounter{tocdepth}{2}
%\tableofcontents
\pagenumbering{arabic}

\section{Introduction}
For the second project for the course Statistical Data Analysis we are once again worikng with the
dataset of spectral data of four cultivars of canteloupe melons. 

Groups we work in this report with were also again chosen randomly, with 50 observations drawn from each of the groups. Below in INSERT REFERENCE we see a figure of 
spectral plots of each of the groups. We see that groups 1, 2 and 4 differ only slightly in lower wavelength numbers with group 4 having more variability there,
but are quite similar in their behaviour in higher frequencies. Group 3 is the one that clearly stands out; 
a part of it resembles the behaviour of other three groups, while some observations form a different pattern. 

We predict that this part of group three will form a 
separate cluster. Another will possibly be formed by lower wavelengths of group 4 -- other observations seem to be too similar to form different clusters.

\section{K-medoids clustering}

\section{Hierarchical clustering}

\section{Conclusion}

\end{document}